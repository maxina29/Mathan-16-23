\documentclass[a4paper,10pt]{article}

\usepackage[T2A]{fontenc}
\usepackage[utf8]{inputenc}
\usepackage[english, russian]{babel}

\usepackage{xcolor}
\usepackage{hyperref}
\usepackage{enumitem}

\definecolor{linkcolor}{HTML}{000000}
\definecolor{urlcolor}{HTML}{000000}

\hypersetup{pdfstartview=FitH, linkcolor=linkcolor,
urlcolor=urlcolor, colorlinks=true}

\usepackage{amssymb,amsthm,amsmath}

\parindent=0.5cm
\parskip=0.1cm

\tolerance=400

\binoppenalty=10000
\relpenalty=10000

\theoremstyle{definition}

\renewenvironment{text}{\par}{\par}

\newcommand{\intl}{\mathop{\int}\limits}
\newcommand{\liml}{\mathop{\lim}\limits}
\newcommand{\suml}{\mathop{\sum}\limits}
\newcommand{\supl}{\mathop{\sup}\limits}
\newcommand{\infl}{\mathop{\inf}\limits}
\newcommand{\longrightarrowl}{\mathop{\longrightarrow}\limits}

\newcommand{\Bo}{{\raisebox{0.2ex}{$\stackrel{\circ}{B}$}}{}}

\renewcommand{\le}{\leqslant}
\renewcommand{\ge}{\geqslant}

\def\N{{\mathbb N}}
\def\Z{{\mathbb Z}}
\def\R{{\mathbb R}}
\def\Q{{\mathbb Q}}
\def\C{{\mathbb C}}

\renewcommand{\Re}{\operatorname{Re}}
\renewcommand{\Im}{\operatorname{Im}}

\def\arsh{\operatorname{arsh}}
\def\arch{\operatorname{arch}}
\def\arth{\operatorname{arth}}
\def\arcth{\operatorname{arcth}}


\def\bydef{\operatorname{def}}

\usepackage{jeolm}
\usepackage{jeolm-groups}
\usepackage{thm}

%\usepackage{anyfontsize}
%\usepackage{upgreek}
\AtBeginDocument{\swapvar{phi}\swapvar{epsilon}\swapvar{nothing}}
\AtBeginDocument{\swapvar[up]{phi}\swapvar[up]{epsilon}}
\usepackage{parskip}
%\pagestyle{empty}

\newcommand*{\hm}[1]{#1\nobreak\discretionary{}
	{\hbox{$\mathsurround=0pt #1$}}{}} % Перенос знаков в формулах (по Львовскому)

\usepackage{geometry}
\geometry{a5paper,portrait,vmargin={2em,2em},hmargin={2em,2em}}

%\usepackage{pgfpages}
%\pgfpagesuselayout{resize to}[a4paper]

\def\jeolmsubject{Математический анализ}
\def\jeolmlector{Садовничая Инна Викторовна}
\def\jeolmfaculty{ФКИ МГУ, 2 семестр}
\def\jeolmurl{Место для вашей рекламы}
\def\jeolmtex{Кириил}

%\AtBeginDocument{\fontsize{9.00}{10.80}\selectfont}
% (по умолчанию 10.00 и 12.00 соответственно)

\def\qedsymbol{$\blacksquare$}
\def\d{\partial}
\newcommand{\drobb}[2]{\ensuremath{{}^{#1}\!/_{#2}}}
\usepackage{ upgreek }
\def\sigmaf{\sigma_{\!\!{}_f}}
\def\hhline{\noindent\rule{\textwidth}{0.5pt}}

\makeatletter
\renewcommand*\env@matrix[1][\arraystretch]{%
	\edef\arraystretch{#1}%
	\hskip -\arraycolsep
	\let\@ifnextchar\new@ifnextchar
	\array{*\c@MaxMatrixCols c}}
\makeatother

\def\grad{\mathop{\operatorname{g}\!\vec{\operatorname{ra}}\!\operatorname{d}}}
\def\div{\mathop{\operatorname{div}}}
\def\rot{\operatorname{\mathop{rot}}}
\usepackage{wrapfig}
\usepackage{asymptote}


\def\jeolmlectnum{17}
\def\jeolmdate{11 апреля 2019}
\def\jeolmlectionname{Сведение двойного интеграла к повторному. Тройные интегралы}
\def\jeolmtex{Ма$\xi$м}

\newcommand{\RNumb}[1]{\uppercase\expandafter{\romannumeral #1\relax}}

\begin{document}
	
	\jeolmnewheader
	
	\Large\RNumb{1}.\normalsize\quad Пусть функция $f$ интегрируема на прямоугольнике $R = [a, b] \times [c, d]$.
	
	\begin{thm}
		Если $\forall x \in [a, b]$ существует $I(x) = \int\limits_c^d f(x, y) \, dy$, то функция $I(x)$ интегрируема на $[a, b]$ и по определению $\int\limits_a^b \left( \int\limits_c^d f(x, y) \, dy \right) dx$ называется \textit{повторным интегралом}; при этом $\iint\limits_R f(x, y) \, dxdy = \int\limits_a^b \left( \int\limits_c^d f(x, y) \, dy \right) dx$.
	\end{thm}
	
	\begin{proof}
		Пусть $a = x_0 < x_1 < \ldots < x_n =b$, $c = y_0 < y_1 < \ldots < y_m = d$, $R_{k\ell} = [x_{k - 1}, x_{k}] \times [y_{\ell - 1}, y_\ell]$, $\Delta x_k = x_k - x_{k - 1}$, $\Delta y_\ell = y_{\ell} - y_{\ell - 1}$, $\Delta_{k\ell} = \sqrt{(\Delta x_k)^2 + (\Delta y_\ell)^2}$, $\Delta_T = \max\limits_{\begin{smallmatrix} 1 \le k \le n \\ 1 \le \ell \le m \end{smallmatrix}}$.
		
		Обозначим $m_{k\ell} = \inf\limits_{R_{k\ell}} f(x, y)$; $M_{k\ell} = \sup\limits_{R_{k\ell}} f(x, y)$. Пусть $\xi_k \in [x_{k - 1}, x_k]$ --- произвольная точка. Очевидно, что $m_{k\ell} \le f(\xi_k, y) \le M_{k\ell} ~ \forall y \in [y_{\ell - 1}, y_\ell]$.
		
		Проинтегрируем это неравенство по $[y_{\ell - 1}, y_\ell] :$
		$$m_{k\ell} \cdot \Delta y_\ell \le \int\limits_{y_{\ell - 1}}^{y_\ell} f(\xi_k, y) \, dy \le M_{k\ell} \cdot \Delta y_\ell$$
		$$\Rightarrow \sum\limits_{\ell = 1}^m m_{k\ell} \cdot \Delta y_\ell \le \underbrace{\int\limits_{c}^{d} f(\xi_k, y) \, dy}_{I(\xi_k)} \le \sum\limits_{\ell = 1}^m M_{k\ell} \cdot \Delta y_\ell$$
		Умножим последнее неравенство на $\Delta x_k$ и просуммируем по $k$ от $1$ до $n$.
		$$s(T) = \sum\limits_{k = 1}^n \sum\limits_{\ell = 1}^m m_{k\ell} \cdot \Delta x_k \Delta y_\ell \le \!\!\!\!\!\!\!\!\! \underbrace{\sum\limits_{k = 1}^n I(\xi_k) \cdot \Delta x_k}_{\begin{smallmatrix}
		\textup{интегральная сумма для} \\
		\textup{функции } I(x) \textup{ на } [a, b]
		\end{smallmatrix}} \!\!\!\!\!\!\!\!\! \le \sum\limits_{k = 1}^n \sum\limits_{\ell = 1}^m M_{k\ell} \cdot \Delta x_k \Delta y_\ell = S(T)$$
		Заметим, что при $\Delta_T \to 0 : s(T) \to \iint\limits_R f(x, y) \, dxdy$ и $S(T) \to \iint\limits_R f(x, y) \, dxdy \Rightarrow$ и средняя часть сходится к этому же значению.
		
		Получим, что при стремлении $\max\limits_{1 \le k \le n} \Delta x_k \to 0 :$ интегральная сумма для функции $I(x)$ на $[a, b]$ стремится к одному и тому же числу при любом выборе точек $\xi_k \Rightarrow$ функция $I(x) \in \mathcal{R}[a, b]$ и $\int\limits_a^b I(x) \, dx = \iint\limits_R f(x, y) \, dxdy$
	\end{proof}
	
	\Large\RNumb{2}.\normalsize\quad Пусть теперь $f \in \mathcal{R}(D)$, где $D$ --- ограниченная квадрируемая область.
	
	\begin{thm}
		Пусть область $D$ такова, что её проекция на $Ox$ есть отрезок $[a, b]; ~ \forall x_0 \in [a, b]$ прямая $x = x_0$ пересекает границу $\d D$ либо по отрезку $[y_1(x_0), y_2(x_0)]$, либо не более, чем в двух точках: $y_1(x_0), y_2(x_0), y_1(x_0) \le y_2(x_0)$. Если $\forall x \in [a, b]$ функция $f(x, y)$ интегрируема на отрезке $[y_1(x), y_2(x)]$, то существует  $$\int\limits_a^b \left( \int\limits_{y_1(x)}^{y_2(x)} f(x, y) \, dy \right) dx$$ причём этот повторный интеграл совпадает с двойным интегралом $\iint\limits_D f(x, y) \, dxdy$
	\end{thm}
	
	\begin{proof}
		Возьмём прямоугольник $R = [a, b] \times [c, d]$, такой, что $\overline{D} \subset R$. Положим $F(x, y) = \begin{cases} f(x, y), &(x, y) \in \overline{D} \\ 0, &(x, y) \in \mathcal{R} \backslash \overline{D} \end{cases}$. Так как $f \in \mathcal{R}(D)$, то по определению $F$ интегрируема на прямоугольнике $R$, причём $\iint\limits_D f(x, y) \, dxdy \mathop{=}\limits_{\operatorname{def}} \iint\limits_R F(x, y) \, dxdy =\\= \textup{(т. 1)} = \int\limits_a^b \left( \int\limits_c^d F(x, y) \, dy \right) dx = \int\limits_a^b \left( \int\limits_{y_1(x)}^{y_2(x)} \underbrace{F(x, y)}_{f(x, y) \textup{ в } D} dy \right) dx = \int\limits_a^b \left( \int\limits_{y_1(x)}^{y_2(x)} f(x, y) \, dy \right) dx$
	\end{proof}
	
	\begin{note}
		Можно доказать теоремы, аналогичные т. 1 и т. 2 для повторных интегралов $\int\limits_c^d \left( \int\limits_{x_1(y)}^{x_2(y)} f(x, y) \, dx \right) dy$
	\end{note}
	%дописать
	
% thm       Теорема
% thm*      Теорема (б/д)
% nthm      Теорема (именная)
% lem       Лемма
% defn      Определение
% prop      Утверждение
% note      Замечание
% cor       Следствие
% exmp      Пример
% xca       Упражнение
% Нумерация во всех свойствах через \item
% props     Свойства
% nprops    Свойства (с названием)
% props*    Свойства (б/д)
% proof     Доказательство
% proof*    Доказательство без точки и конца доказательства
% proofs*	Доказательства без точки и конца доказательства
% noproof	Без доказательства
% solution  Решение

\end{document}