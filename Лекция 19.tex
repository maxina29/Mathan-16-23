\documentclass[a4paper,10pt]{article}

\usepackage[T2A]{fontenc}
\usepackage[utf8]{inputenc}
\usepackage[english, russian]{babel}

\usepackage{xcolor}
\usepackage{hyperref}
\usepackage{enumitem}

\definecolor{linkcolor}{HTML}{000000}
\definecolor{urlcolor}{HTML}{000000}

\hypersetup{pdfstartview=FitH, linkcolor=linkcolor,
urlcolor=urlcolor, colorlinks=true}

\usepackage{amssymb,amsthm,amsmath}

\parindent=0.5cm
\parskip=0.1cm

\tolerance=400

\binoppenalty=10000
\relpenalty=10000

\theoremstyle{definition}

\renewenvironment{text}{\par}{\par}

\newcommand{\intl}{\mathop{\int}\limits}
\newcommand{\liml}{\mathop{\lim}\limits}
\newcommand{\suml}{\mathop{\sum}\limits}
\newcommand{\supl}{\mathop{\sup}\limits}
\newcommand{\infl}{\mathop{\inf}\limits}
\newcommand{\longrightarrowl}{\mathop{\longrightarrow}\limits}

\newcommand{\Bo}{{\raisebox{0.2ex}{$\stackrel{\circ}{B}$}}{}}

\renewcommand{\le}{\leqslant}
\renewcommand{\ge}{\geqslant}

\def\N{{\mathbb N}}
\def\Z{{\mathbb Z}}
\def\R{{\mathbb R}}
\def\Q{{\mathbb Q}}
\def\C{{\mathbb C}}

\renewcommand{\Re}{\operatorname{Re}}
\renewcommand{\Im}{\operatorname{Im}}

\def\arsh{\operatorname{arsh}}
\def\arch{\operatorname{arch}}
\def\arth{\operatorname{arth}}
\def\arcth{\operatorname{arcth}}


\def\bydef{\operatorname{def}}

\usepackage{jeolm}
\usepackage{jeolm-groups}
\usepackage{thm}

%\usepackage{anyfontsize}
%\usepackage{upgreek}
\AtBeginDocument{\swapvar{phi}\swapvar{epsilon}\swapvar{nothing}}
\AtBeginDocument{\swapvar[up]{phi}\swapvar[up]{epsilon}}
\usepackage{parskip}
%\pagestyle{empty}

\newcommand*{\hm}[1]{#1\nobreak\discretionary{}
	{\hbox{$\mathsurround=0pt #1$}}{}} % Перенос знаков в формулах (по Львовскому)

\usepackage{geometry}
\geometry{a5paper,portrait,vmargin={2em,2em},hmargin={2em,2em}}

%\usepackage{pgfpages}
%\pgfpagesuselayout{resize to}[a4paper]

\def\jeolmsubject{Математический анализ}
\def\jeolmlector{Садовничая Инна Викторовна}
\def\jeolmfaculty{ФКИ МГУ, 2 семестр}
\def\jeolmurl{Место для вашей рекламы}
\def\jeolmtex{Кириил}

%\AtBeginDocument{\fontsize{9.00}{10.80}\selectfont}
% (по умолчанию 10.00 и 12.00 соответственно)

\def\qedsymbol{$\blacksquare$}
\def\d{\partial}
\newcommand{\drobb}[2]{\ensuremath{{}^{#1}\!/_{#2}}}
\usepackage{ upgreek }
\def\sigmaf{\sigma_{\!\!{}_f}}
\def\hhline{\noindent\rule{\textwidth}{0.5pt}}

\makeatletter
\renewcommand*\env@matrix[1][\arraystretch]{%
	\edef\arraystretch{#1}%
	\hskip -\arraycolsep
	\let\@ifnextchar\new@ifnextchar
	\array{*\c@MaxMatrixCols c}}
\makeatother

\def\grad{\mathop{\operatorname{g}\!\vec{\operatorname{ra}}\!\operatorname{d}}}
\def\div{\mathop{\operatorname{div}}}
\def\rot{\operatorname{\mathop{rot}}}
\usepackage{wrapfig}
\usepackage{asymptote}


\def\jeolmlectnum{19}
\def\jeolmdate{18 апреля 2019}
\def\jeolmlectionname{Криволинейные интегралы}
\def\jeolmtex{}

\begin{document}
	
	\jeolmnewheader
	
Пусть $L = \left\{ (x, y) \in \mathbb{R}^2 \mid x = \phi(t), ~ y = \psi(t), ~ \alpha \le t \le \beta; \phi, ~ \psi \in C\mspace{1mu}[\alpha, \beta] \right\}$ --- простая спрямляемая кривая. $A = (\phi(\alpha), \, \psi(\alpha)), ~ B = (\phi(\beta), \, \psi(\beta))$. Будем писать $L$~или~$AB$. Пусть функции $f(x, y), ~ P(x, y), ~ Q(x, y)$ непрерывны на кривой $L$.

\begin{defn}
Пусть $T = \left\{t_0, t_1, \dots , t_n \right\}$ --- разбиение отрезка $[\alpha, \beta]$. Обозначим $x_k = \phi(t_k), ~ y_k = \psi(t_k), ~ k = 0, \dots, n; ~ \Delta x_k = x_k - x_{k-1}, ~ \Delta y_k = y_k - y_{k - 1};$ \mbox{$\Delta_{T}$ --- диаметр} разбиения отрезка $T$; $M_k(x_k, y_k)$ --- точка на кривой $L$.
\end{defn}

% Для справки:
% Палочка при задании множества --- \mid
% \leq \geq => \le \ge
% Вместо \mathbb{R} можно писать \R

% thm       Теорема
% thm*      Теорема (б/д)
% nthm      Теорема (именная)
% lem       Лемма
% defn      Определение
% prop      Утверждение
% note      Замечание
% cor       Следствие
% exmp      Пример
% xca       Упражнение
% Нумерация во всех свойствах через \item
% props     Свойства
% nprops    Свойства (с названием)
% props*    Свойства (б/д)
% proof     Доказательство
% proof*    Доказательство без точки и конца доказательства
% proofs*	Доказательства без точки и конца доказательства
% noproof	Без доказательства
% solution  Решение

\end{document}