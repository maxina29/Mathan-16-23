% заТеХано
\documentclass[a4paper,10pt]{article}

\usepackage[T2A]{fontenc}
\usepackage[utf8]{inputenc}
\usepackage[english, russian]{babel}

\usepackage{xcolor}
\usepackage{hyperref}
\usepackage{enumitem}

\definecolor{linkcolor}{HTML}{000000}
\definecolor{urlcolor}{HTML}{000000}

\hypersetup{pdfstartview=FitH, linkcolor=linkcolor,
urlcolor=urlcolor, colorlinks=true}

\usepackage{amssymb,amsthm,amsmath}

\parindent=0.5cm
\parskip=0.1cm

\tolerance=400

\binoppenalty=10000
\relpenalty=10000

\theoremstyle{definition}

\renewenvironment{text}{\par}{\par}

\newcommand{\intl}{\mathop{\int}\limits}
\newcommand{\liml}{\mathop{\lim}\limits}
\newcommand{\suml}{\mathop{\sum}\limits}
\newcommand{\supl}{\mathop{\sup}\limits}
\newcommand{\infl}{\mathop{\inf}\limits}
\newcommand{\longrightarrowl}{\mathop{\longrightarrow}\limits}

\newcommand{\Bo}{{\raisebox{0.2ex}{$\stackrel{\circ}{B}$}}{}}

\renewcommand{\le}{\leqslant}
\renewcommand{\ge}{\geqslant}

\def\N{{\mathbb N}}
\def\Z{{\mathbb Z}}
\def\R{{\mathbb R}}
\def\Q{{\mathbb Q}}
\def\C{{\mathbb C}}

\renewcommand{\Re}{\operatorname{Re}}
\renewcommand{\Im}{\operatorname{Im}}

\def\arsh{\operatorname{arsh}}
\def\arch{\operatorname{arch}}
\def\arth{\operatorname{arth}}
\def\arcth{\operatorname{arcth}}


\def\bydef{\operatorname{def}}

\usepackage{jeolm}
\usepackage{jeolm-groups}
\usepackage{thm}

%\usepackage{anyfontsize}
%\usepackage{upgreek}
\AtBeginDocument{\swapvar{phi}\swapvar{epsilon}\swapvar{nothing}}
\AtBeginDocument{\swapvar[up]{phi}\swapvar[up]{epsilon}}
\usepackage{parskip}
%\pagestyle{empty}

\newcommand*{\hm}[1]{#1\nobreak\discretionary{}
	{\hbox{$\mathsurround=0pt #1$}}{}} % Перенос знаков в формулах (по Львовскому)

\usepackage{geometry}
\geometry{a5paper,portrait,vmargin={2em,2em},hmargin={2em,2em}}

%\usepackage{pgfpages}
%\pgfpagesuselayout{resize to}[a4paper]

\def\jeolmsubject{Математический анализ}
\def\jeolmlector{Садовничая Инна Викторовна}
\def\jeolmfaculty{ФКИ МГУ, 2 семестр}
\def\jeolmurl{Место для вашей рекламы}
\def\jeolmtex{Кириил}

%\AtBeginDocument{\fontsize{9.00}{10.80}\selectfont}
% (по умолчанию 10.00 и 12.00 соответственно)

\def\qedsymbol{$\blacksquare$}
\def\d{\partial}
\newcommand{\drobb}[2]{\ensuremath{{}^{#1}\!/_{#2}}}
\usepackage{ upgreek }
\def\sigmaf{\sigma_{\!\!{}_f}}
\def\hhline{\noindent\rule{\textwidth}{0.5pt}}

\makeatletter
\renewcommand*\env@matrix[1][\arraystretch]{%
	\edef\arraystretch{#1}%
	\hskip -\arraycolsep
	\let\@ifnextchar\new@ifnextchar
	\array{*\c@MaxMatrixCols c}}
\makeatother

\def\grad{\mathop{\operatorname{g}\!\vec{\operatorname{ra}}\!\operatorname{d}}}
\def\div{\mathop{\operatorname{div}}}
\def\rot{\operatorname{\mathop{rot}}}
\usepackage{wrapfig}
\usepackage{asymptote}


\def\jeolmlectnum{21}
\def\jeolmdate{25 апреля 2019}
\def\jeolmlectionname{Площадь поверхности. Понятие поверхностного интеграла.}
\def\jeolmtex{Ма$\xi$м}
\usepackage{mathrsfs}
\begin{document}
	
	\jeolmnewheader
	
	Пусть гладкая ограниченная двусторонняя поверхность $\Phi$ без особых точек задана уравнениями
	\begin{equation}\label{eq:1}
		x = x(u, v), y=y(u, v), z = z(u, v)
	\end{equation}
	$(u, v) \in G, G \subset \R^2$ --- замкнутое связное квадрируемое множество. 
	
	В силу леммы 2 предыдущей лекции $\exists \delta > 0$, такое, что поверхность $\Phi$ можно разбить конечным числом гладких кривых на части $\Phi_i$ размера меньше $\delta$ так, чтобы каждая часть однозначно проектировалась на касательную плоскость, проведенную в любой точке $\Phi_i$.
	
	\begin{defn}
		Число $\Delta =$ максимальному размеру участков $\Phi_i$ называется \textit{диаметром разбиения}.
	\end{defn}
	
	\begin{defn}
		Выберем на каждом участке $\Phi_i$ точку $M_i$ произвольным образом и спроецируем $\Phi_i$ на касательную плоскость в точке $M_i$.

		Пусть $\sigma_i$ --- площадь получившейся проекции. Число $\sigma$ называется \textit{пределом сумм} $\sum\limits_i \sigma_i$ при $\Delta \to 0$, если $\forall \epsilon > 0 ~ \exists \delta = \delta (\epsilon) > 0$ такое, что любое разбиение с диаметром $\Delta < \delta$, независимо от выбора точек $M_i : \left|\sigma - \sum\limits_i \sigma_i\right| < \epsilon$.
	
		Поверхность $\Phi$, для которой такой предел существует, называется \textit{квадрируемой}; число $\sigma$ называется \textit{площадью поверхности} $\Phi$.
	\end{defn}

	% сапог Шварца
	
	\begin{thm}\label{thm:1}
		Пусть $\Phi$ --- гладкая двусторонняя ограниченная поверхность без особых точек, задана уравнениями \eqref{eq:1} на $G$; $G$ --- замкнутое линейно связное квадрируемое множество.
		
		Тогда $\Phi$ --- квадрируемая поверхность, причем ее площадь $\sigma$ вычисляется по формуле
		\begin{equation}\label{eq:2}
			\sigma = \iint\limits_G \left| \left[ \dfrac{\d \vec{r}}{\d u}, \dfrac{\d \vec{r}}{\d v} \right] \right| dudv
		\end{equation}
		$\vec{r}\,(u, v) = \{x(u, v), y(u, v), z(u, v)\}$
	\end{thm}

	\begin{proof}
		Обозначим через $J$ величину интеграла в правой части \eqref{eq:2}.
		
		Почкольку подынтегральная функция непрерывна (т.к. $\Phi$ --- гладкая поверхность), то интеграл существует. Пусть $\epsilon > 0$. В силу лемм $2$ и $3$ предыдущей лекции $\exists \delta = \delta(\epsilon) > 0$, такое, что:
		\begin{enumerate}
			\item Любая часть поверхности размера меньше $\delta$ однозначно проецируется на касательную плоскость, проведённую в любой точке этой части.
			\item Если размер $\Phi_i$ меньше $\delta$, то $\forall M_1, M_2 \in \Phi_i ~ \cos \gamma = 1 - \alpha$, где \\ $\gamma = \angle(\vec{n}(M_1), \vec{n}(M_2))$; $0 \le \alpha < \min\{1, \drobb{\epsilon}{J}\}$.
		\end{enumerate}
		Пусть есть разбиение $\Phi$ на участке $\Phi_i$, $\Delta < \delta$. Выберем произвольным образом $M_i \in \Phi_i$, спроецируем $\Phi_i$ на касательную плоскость в точке $M_i$ и вычислим площадь $\sigma_i$ полученной проекции. Введём систему координат так, чтобы $Oz \parallel \vec{n}(M_i)$; $Ox$ и $Oy$ лежали в касательной плоскости. Тогда $$\left[ \dfrac{\d \vec{r}}{\d u}, \dfrac{\d \vec{r}}{\d v} \right] = \left| 
		\begin{matrix}
		\vec{i} & \vec{j} & \vec{k} \\[.1em]
		\dfrac{\d x}{\d u} & \dfrac{\d y}{\d u} & \dfrac{\d z}{\d u} \\[.7em]
		\dfrac{\d x}{\d v} & \dfrac{\d y}{\d v} & \dfrac{\d z}{\d v}
		\end{matrix} 
		\right| = \{A, B, C\},$$
		где $A = \left| \begin{matrix} y_u' & z_u' \\ y_v' & z_v' \end{matrix} \right|$, $B = \left| \begin{matrix} z_u' & x_u' \\ z_v' & x_v' \end{matrix} \right|$, $C = \left| \begin{matrix} x_u' & y_u' \\ x_v' & y_v' \end{matrix} \right|$.
		
		Заметим, что если $\gamma_{{}_M} = \angle(\vec{n}(M), \vec{n}(M_i)), M \in \Phi_i$, то  
		$$\cos \gamma_{{}_M} = \dfrac{(\vec{n}(M), \{0, 0, 1\})}{|\vec{n}(M)|} \hm{=} \left. \dfrac{C}{|[r_u', r_v']|} \right|_M$$
		Поскольку $\cos \gamma_{{}_M} > 0\Rightarrow C > 0$.
		
		Пусть $\widetilde{G}_i$ --- проекция $\Phi_i$ на касательную плоскость в точке $M_i$. Тогда её площадь 
		\begin{multline*}
		\sigma_i = \iint\limits_{\widetilde{G}_i} 1\,dxdy = \left| \begin{matrix} x = x(u, v) \\ y = y(u, v) \end{matrix} \right| = \iint\limits_{G_i} \bigg| \underbrace{\dfrac{D(x, y)}{D(u, v)}}_{C>0} \bigg| \,dudv 
		= \\ =
		\iint\limits_{G_i} \cos \gamma_{{}_M} \cdot \Big| [r_u', r_v'] \Big| \,dudv = 
		\left( \begin{matrix} \textup{1-ая теорема о среднем} \\ \cos \gamma_{{}_M} \textup{ непрерывен} \end{matrix} \right) 
		= \\ = 
		\cos \gamma_{{}_{M^*}} \cdot \iint\limits_{G_i} \Big| [r_u', r_v'] \Big| \,dudv = (1 - \alpha) \cdot \iint\limits_{G_i} \Big| [r_u', r_v'] \Big| \,dudv
		\end{multline*}
		Отсюда $\sum\limits_i \sigma_i = \underbrace{\sum\limits_i \iint\limits_{G_i} \Big| [r_u', r_v'] \Big| \,dudv}_{J} - \alpha \cdot J \Rightarrow$ \\
		$\left| J - \sum\limits_i \sigma_i \right| = \alpha \cdot J < (\textup{в силу выбора }\alpha) < \dfrac{\epsilon}{J} \cdot J = \epsilon = \lim\limits_{\Delta \to 0} \sum\limits_i \sigma_i \Rightarrow J = \sigma$
	\end{proof}
	
	\begin{note}
		Обозначим $E = \left| \dfrac{\d \vec{r}}{\d u} \right|^2$; $D = \left| \dfrac{\d \vec{r}}{\d v} \right|^2$; $F = \left( \dfrac{\d \vec{r}}{\d u}, \dfrac{\d \vec{r}}{\d v} \right)$. 
		
		Поскольку для любых векторов $\vec{a}, \vec{b}$: $$\underbrace{[\vec{a}, \vec{b}]^2}_{|\vec{a}|^2 \cdot |\vec{b}|^2 \cdot \sin^2 \alpha} + \underbrace{(\vec{a}, \vec{b})^2}_{|\vec{a}|^2 \cdot |\vec{b}|^2 \cdot \cos^2 \alpha} = |\vec{a}|^2 + |\vec{b}|^2  \Rightarrow \left| \left[ \dfrac{\d \vec{r}}{\d u}, \dfrac{\d \vec{r}}{\d v} \right] \right| = \sqrt{ED - F^2}$$
		Тогда площадь поверхности вычисляется по формуле 
		\begin{equation}\label{eq:3}
			\sigma = \iint\limits_G \sqrt{ED - F^2}\, dudv
		\end{equation}
	\end{note}
	
	\begin{exmp}
		Вычислим площадь поверхности сферы радиуса $R$.
	\end{exmp}
	\begin{solution}
		Рассмотрим полусферу: $\{(x, y, z) \mid x^2 + y^2 + z^2 = R^2, z \ge 0\}$. Запишем параметризацию в сферических координатах: $\begin{cases}
		x = R \cos \phi \sin \psi; \\
		y = R \sin \phi \sin \psi; \\
		z = R \cos \psi.
		\end{cases}$ \\
		$0 \le \phi < 2\pi$, $0 \le \psi \le \drobb{\pi}{2}$
		
		$\dfrac{\d \vec{r}}{\d \phi} = \{-R \sin \phi \sin \psi, R \cos \phi \sin \psi, 0\} \Rightarrow E = R^2 \sin^2 \psi$
		
		$\dfrac{\d \vec{r}}{\d \psi} = \{R \cos \phi \cos \psi, R \sin \phi \cos \psi, -R \sin \psi \} \Rightarrow D = R^2 \cos^2 \psi + R^2 \sin^2 \psi = R^2$
		
		$F = R^2 \cdot (0) = 0 \Rightarrow \sqrt{ED - F^2} = R^2 \sin \psi \Rightarrow$ 
		
		$\dfrac{1}{2}S_{\textup{пов. сф.}} = \iiint\limits_G \sqrt{ED - F^2}\, d\phi d\psi = \int\limits_0^{2\pi} \int\limits_0^{\drobb{\pi}{2}} R^2 \sin \psi\, d\psi d\phi = R^2 \cdot 2\pi \cdot (-\cos \psi) \big|_{0}^{\drobb{\pi}{2}} = 2\pi R^2$
		
		$\Rightarrow \fbox{$S_{\textup{пов. сферы}} = 4\pi R^2$}$
	\end{solution}
	
	\begin{defn}
		Пусть поверхность $\Phi$ удовлетворяет условиям теоремы 1, на ней определены непрерывные функции $f(x, y, z)$, $P(x, y, z)$, $Q(x, y, z)$, $R(x, y, z)$. Возьмём разбиение поверхности $\Phi$ на участки $\Phi_i$ гладкими кривыми; размеры всех $\Phi_i$ меньше $\delta$, где $\delta$ --- из леммы 2 предыдущей лекции.
		
		Возьмём произвольные $M_i \in \Phi_i$ и составим суммы: $$\Sigma_1 = \sum\limits_i f(M_i) \cdot \sigma_i$$ ($\sigma_i$ --- площадь проекции $\Phi_i$ на касательную плоскость в точке $M_i$)
		
		Пусть $\underbrace{\vec{n}(M_i)}_{\begin{smallmatrix} \textup{единичная} \\ \textup{нормаль} \end{smallmatrix}} \!\!\!\! = \{\cos X_i, \cos Y_i, \cos Z_i\}$
		$$\Sigma_2 = \sum\limits_i P(M_i) \cdot \cos X_i \cdot \sigma_i;$$
		$$\Sigma_3 = \sum\limits_i Q(M_i) \cdot \cos Y_i \cdot \sigma_i;$$
		$$\Sigma_4 = \sum\limits_i R(M_i) \cdot \cos Z_i \cdot \sigma_i.$$
	\end{defn}
	
	\begin{defn}
		\textit{Поверхностным интегралом 1-го рода} называется число $I_1$, такое, что $\forall \epsilon > 0 ~ \exists \delta(\epsilon) > 0$, такое, что для любого разбиения с диаметром $\Delta < \delta(\epsilon)$, для любого выбора точек $M_i : |\Sigma_1 - I_1| < \epsilon$.
		
		Обозначение: $\iint\limits_\Phi f(M) \!\!\!\!\! \underbrace{d\sigma}_{\begin{smallmatrix} \textup{элемент} \\ \textup{площади} \end{smallmatrix}}$
		
		\textit{Поверхностными интегралами 2-го рода} называются числа $I_k$, $k = 2, 3, 4$, такие, что $\forall \epsilon > 0 ~ \exists \delta(\epsilon) > 0$, такое, что для любого разбиения с диаметром $\Delta < \delta(\epsilon)$, для любого выбора точек $M_i : |\Sigma_k - I_k| < \epsilon$. 
		
		Обозначение: $\iint\limits_\Phi P(M) \cdot \cos X \, d\sigma$ и \!т.\,д.
		
		Сумма $I_2 + I_3 + I_4$ называется \textit{общим поверхностным интегралом 2-го рода}.
		
		Обозначение: $\iint\limits_\Phi (\vec{A}, \vec{n}) \, d\sigma$, где $\vec{A} = \{P, Q, R\}$, $\vec{n} = \{\cos X, \cos Y, \cos Z \}$ --- вектор единичной нормали.
	\end{defn}
	
	\textbf{\textsc{Свойства:}}
	\begin{enumerate}[label = \arabic*)]
		\item Интеграл 1-го рода не зависит от направления нормали; интегралы 2-го рода меняют знак при смене ориентации вектора нормали;
		\item Интеграл 1-го рода и общий интеграл 2-го рода инвариантны относительно выбора системы координат;
		\item Интеграл 1-го рода --- масса нагруженной поверхности $\Phi$ с плотностью распределения масс $f(x, y, z)$. \\ 
		Общий интеграл 2-го рода --- поток вектора $A$ через поверхность $\Phi$.
	\end{enumerate}
	% заТеХано
\end{document}