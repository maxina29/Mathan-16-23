% заТеХано
\documentclass[a4paper,10pt]{article}

\usepackage[T2A]{fontenc}
\usepackage[utf8]{inputenc}
\usepackage[english, russian]{babel}

\usepackage{xcolor}
\usepackage{hyperref}
\usepackage{enumitem}

\definecolor{linkcolor}{HTML}{000000}
\definecolor{urlcolor}{HTML}{000000}

\hypersetup{pdfstartview=FitH, linkcolor=linkcolor,
urlcolor=urlcolor, colorlinks=true}

\usepackage{amssymb,amsthm,amsmath}

\parindent=0.5cm
\parskip=0.1cm

\tolerance=400

\binoppenalty=10000
\relpenalty=10000

\theoremstyle{definition}

\renewenvironment{text}{\par}{\par}

\newcommand{\intl}{\mathop{\int}\limits}
\newcommand{\liml}{\mathop{\lim}\limits}
\newcommand{\suml}{\mathop{\sum}\limits}
\newcommand{\supl}{\mathop{\sup}\limits}
\newcommand{\infl}{\mathop{\inf}\limits}
\newcommand{\longrightarrowl}{\mathop{\longrightarrow}\limits}

\newcommand{\Bo}{{\raisebox{0.2ex}{$\stackrel{\circ}{B}$}}{}}

\renewcommand{\le}{\leqslant}
\renewcommand{\ge}{\geqslant}

\def\N{{\mathbb N}}
\def\Z{{\mathbb Z}}
\def\R{{\mathbb R}}
\def\Q{{\mathbb Q}}
\def\C{{\mathbb C}}

\renewcommand{\Re}{\operatorname{Re}}
\renewcommand{\Im}{\operatorname{Im}}

\def\arsh{\operatorname{arsh}}
\def\arch{\operatorname{arch}}
\def\arth{\operatorname{arth}}
\def\arcth{\operatorname{arcth}}


\def\bydef{\operatorname{def}}

\usepackage{jeolm}
\usepackage{jeolm-groups}
\usepackage{thm}

%\usepackage{anyfontsize}
%\usepackage{upgreek}
\AtBeginDocument{\swapvar{phi}\swapvar{epsilon}\swapvar{nothing}}
\AtBeginDocument{\swapvar[up]{phi}\swapvar[up]{epsilon}}
\usepackage{parskip}
%\pagestyle{empty}

\newcommand*{\hm}[1]{#1\nobreak\discretionary{}
	{\hbox{$\mathsurround=0pt #1$}}{}} % Перенос знаков в формулах (по Львовскому)

\usepackage{geometry}
\geometry{a5paper,portrait,vmargin={2em,2em},hmargin={2em,2em}}

%\usepackage{pgfpages}
%\pgfpagesuselayout{resize to}[a4paper]

\def\jeolmsubject{Математический анализ}
\def\jeolmlector{Садовничая Инна Викторовна}
\def\jeolmfaculty{ФКИ МГУ, 2 семестр}
\def\jeolmurl{Место для вашей рекламы}
\def\jeolmtex{Кириил}

%\AtBeginDocument{\fontsize{9.00}{10.80}\selectfont}
% (по умолчанию 10.00 и 12.00 соответственно)

\def\qedsymbol{$\blacksquare$}
\def\d{\partial}
\newcommand{\drobb}[2]{\ensuremath{{}^{#1}\!/_{#2}}}
\usepackage{ upgreek }
\def\sigmaf{\sigma_{\!\!{}_f}}
\def\hhline{\noindent\rule{\textwidth}{0.5pt}}

\makeatletter
\renewcommand*\env@matrix[1][\arraystretch]{%
	\edef\arraystretch{#1}%
	\hskip -\arraycolsep
	\let\@ifnextchar\new@ifnextchar
	\array{*\c@MaxMatrixCols c}}
\makeatother

\def\grad{\mathop{\operatorname{g}\!\vec{\operatorname{ra}}\!\operatorname{d}}}
\def\div{\mathop{\operatorname{div}}}
\def\rot{\operatorname{\mathop{rot}}}
\usepackage{wrapfig}
\usepackage{asymptote}


\def\jeolmlectnum{20}
\def\jeolmdate{22 апреля 2019}
\def\jeolmlectionname{Понятие поверхности в $\R^3$}
\def\jeolmtex{Ма$\xi$м}

\begin{document}
	
	\jeolmnewheader
	
	\begin{defn}
		Отображение $f : G \to G^*$, $G \subset \R^2$, $G^* \subset \R^3$ называется \textit{гомеоморфным}, если оно взаимнооднозначно и последовательность $\{M_n\} \subset G$ фундаментальна $\Leftrightarrow$ фундаментальна последовательность $\{f(M_n)\} \subset G^*$ (это означает, что отображение $f:\overline{G} \to \overline{G^*}$ непрерывно вместе со своим обратным)
	\end{defn}

	\begin{defn}
		Отображение $f : G \to G^*$ называется \textit{локально гомеоморфным}, если у каждой точки $M \in G$ существует окрестность, которая гомеоморфно отображается на свой образ.
	\end{defn}
	
	\begin{defn}\label{def:3}
		Пусть $D \subset \R^2$ --- область (открытое линейно связное множество); $G \subset D$, $G$ --- замкнутое линейное связное множество. \textit{Поверхностью} будем называть множество $\Phi = f(G)$, где $f$ осуществляет локально гомеоморфное отображение области $D$ в $\R^3$.
	\end{defn}

	Пусть $G$ --- множество из определения $3$. Зададим на $G$ функции 
	\begin{equation}\label{eq:1}
		x = x(u, v), y = y(u, v), z = z(u, v)
	\end{equation}
	
	\begin{thm}
		Пусть функции \eqref{eq:1}:
		\begin{enumerate}[label=\arabic*)]
			\item\label{uslovie:1} имеют непрерывные частные производные всюду в области $D \supset G$;
			\item\label{uslovie:2} матрица $A=\left(\begin{matrix}[1.3]
			\frac{\d x}{\d u} & \frac{\d y}{\d u} & \frac{\d z}{\d u} \\
			\frac{\d x}{\d v} & \frac{\d y}{\d v} & \frac{\d z}{\d v}
			\end{matrix}\right)$ всюду в $D$ имеет ранг $2$.
		\end{enumerate}
		Тогда множество $\Phi = \{(x, y, z) \in \R^3 \mid x = x(u, v), y = y(u, v), z = z(u, v); (u, v) \in G \}$ является поверхностью.
	\end{thm}

	\begin{proof}
		Возьмём точку $N_0(u_0, v_0) \in D$; обозначим $x_0 = x(u_0, v_0)$, $y_0 = y(u_0, v_0)$, $z_0 = z(u_0, v_0)$; $M_0=(x_0, y_0, z_0)$. \\
		Матрица $A$ имеет ранг $2$ в точке $N_0 \Rightarrow$ хотя бы один из её миноров второго порядка не равен нулю в этой точке. Пусть, например, $\dfrac{D(x, y)}{D(u, v)}\bigg|_{N_0} \ne 0$. \\
		Рассмотрим систему уравнений: $\begin{cases}
		x(u, v) - x = 0 \\
		y(u, v) - y = 0
		\end{cases}$. \\
		Для этой системы выполнены все условия теоремы о разрешимости системы функциональных уравнений в окрестности точки $M_0$ ($x(u, v)$, $y(u, v)$ имеют непрерывные частные производные $\Rightarrow$ дифференцируемы). 
		Значит, в некоторой окрестности точки $P_0(x_0, y_0)~\exists!$ дифференцируемое решение $\begin{cases}
		u = u(x, y) \\
		v = v(x, y)
		\end{cases}$. \\
		Значит, мы получили гомеоморфное отображение окрестности точки $N_0$ на окрестность точки $P_0$. \\
		Подставим это решение в функцию $z(u, v)$, получим $z(u(x, y), v(x, y)) = \fbox{$\phi(x, y) = z$}$ \\ 
		Получили гомеоморфное отображение окрестности точки $P_0$ на окрестность точки $M_0$. Композиция гомеоморфных отображений является гомеоморфным отображением $\Rightarrow$ мы имеем гомеоморфное отображение окрестности точки точки $N_0$ на окрестность точки $M_0$.
		% дописать
	\end{proof}

	\begin{note}
		Фактически мы показали, что малая окрестность любой точки поверхности однозначно проектируется на одну из координатных плоскостей (в нашем случае --- на плоскость $Oxy$ с помощью функции $\phi$).
	\end{note}
	
	\begin{defn}
		Если поверхность удовлетворяет условию \ref{uslovie:1} теоремы, то она называется \textit{гладкой}; если она удовлетворяет условию \ref{uslovie:2}, то говорят, что она \textit{не имеет особых точек}.
	\end{defn}
	
	Будем рассматривать только гладкие ограниченные поверхности без особых точек.
	
	\begin{defn}
		Обозначим $\vec{r}\,(u, v) = \{x(u, v), y(u, v), z(u, v)\}$. Из условия \ref{uslovie:2} теоремы $\Rightarrow$ векторы $\vec{r}_u\,\!\!\!'$, $\vec{r}_v\,\!\!\!'$ неколлинеарны $\Rightarrow$ они однозначно задают плоскость, которая называется \textit{касательной плоскостью} к поверхности $\Phi$. \\
		Вектор $\vec{n} = \dfrac{[\vec{r}_u\,\!\!\!', \vec{r}_v\,\!\!\!']}{|[\vec{r}_u\,\!\!\!', \vec{r}_v\,\!\!\!']|}$ --- \textit{вектор единичной нормали} к поверхности $\Phi$.
	\end{defn}
	
	Заметим, что из условия \ref{uslovie:1} теоремы $\Rightarrow$ в окрестности каждой точки вектор $\vec{n}$ непрерывно зависит от переменных $(\vec{u}, \vec{v})$. В этом случае говорят, что в каждой точке поверхности \textit{локально существует непрерывное поле нормалей}.
	
	Хотим рассматривать поверхности, для которых непрерывное поле нормалей существует для всей поверхности в целом.
	
	Такие поверхности называются \textit{ориентируемыми} или \textit{двусторонними} (пример односторонней поверхности --- лента Мёбиуса).
	
	В дальнейшем все поверхности --- двусторонние.
	
	\begin{lem}
		Для любой точки $M \subset \Phi$ существует окрестность $\widehat{\Phi}$ этой точки, которая однозначно проектируется на касательную плоскость, проведённую в любой точке этой окрестности (под окрестностью точки $M$ на поверхности понимаем пересечение окрестности этой точки в $\R^3$ с поверхностью).
	\end{lem}
	
	\begin{proof}
		Пусть $M \in \Phi$. Построим окрестность $\widehat{\Phi}$, удовлетворяющую следующим условиям:
		\begin{enumerate}[label=\arabic*)]
			\item угол между $\vec{n}(M)$ (так будем обозначать нормаль в точке $M$) и $\vec{n}(N)$, где $N \in \widehat{\Phi} $, меньше \drobb{\pi}{4} (можем это сделать в силу непрерывности нормали в точке $M$). Тогда угол между $\vec{n}(M_1)$ и $\vec{n}(M_2) ~ \forall M_1, M_2 \in \widehat{\Phi}$ меньше \drobb{\pi}{2}.
			\item Окрестность $\widehat{\Phi}$ однозначно проектируется на одну из координатных плоскостей (см. замечание 1 после теоремы 1), например, на $Oxy$.
		\end{enumerate}
		Предположим, что $\widehat{\Phi}$ не проектируется однозначно на касательную плоскость, проведённую к поверхности в точке $M$. Значит, существуют точки $P, Q \in \widehat{\Phi}$, такие, что хорда $PQ \parallel \vec{n}(M)$. \\
		Проведём через $PQ$ плоскость $\alpha$, параллельную оси $Oz$. Пусть $l=\widehat{\Phi} \cap \alpha$. Тогда по теореме Лагранжа на кривой $L$ существует точка $N$, касательная в которой параллельна $PQ \parallel \vec{n}(M)$. Тогда $\vec{n}(N) \perp \vec{n}(M)$. (?!) с условием \ref{uslovie:1}
	\end{proof}
	
	\begin{defn}
		Будем говорить, что часть $\widehat{\Phi}$ поверхности $\Phi$ \textit{имеет размер меньше} $\delta ~ (\delta > 0)$, если он лежит внутри некоторого шара радиуса \drobb{\delta}{2}.
	\end{defn}
	
	\begin{lem}
		Для данной поверхности $\Phi ~ \exists \delta > 0$, такое, что любая часть $\widehat{\Phi}$ этой поверхности размера меньше $\delta$ однозначно проектируется на одну из координатных плоскостей и на касательную плоскость.
	\end{lem}
	
	\begin{proof}
		Предположим противное: $\forall n \in N ~ \exists$ часть $\Phi_n$ размера меньше \drobb{1}{n}, которая не проектируется либо на координатную плоскость, либо на касательную плоскость в точке $M_n \in \Phi_n$. Из последовательности $\{M_n\}$ выделим сходящуюся подпоследовательность $M_{k_n} \to M_0 \in \Phi$. \\
		Из леммы 1 $\Rightarrow \exists$ окрестность $\widehat{\Phi}$ точки $M_0$, которая однозначно проецируется на координатную плоскость и касательную плоскость. Размер $\Phi_{k_n} < \frac{1}{k_n} \Rightarrow \exists N \in \N$, такое, что $\forall n \ge N : \Phi_{k_n} \subset \widehat{\Phi} \Rightarrow \Phi_{k_n}$ однозначно проецируется. (?!)
	\end{proof}
	
	\begin{lem}
		$\forall \epsilon > 0 ~ \exists \delta = \delta(\epsilon) > 0$, такое, что $\forall$ части $\widehat{\Phi}$ поверхности $\Phi$ размера меньше $\delta :$ угол $\gamma$ между нормалями $\vec{n}(M_1)$ и $\vec{n}(M_2)$, $M_1, M_2 \in \widehat{\Phi}$, удовлетворяет соотношению $\cos \gamma = 1 - \alpha$, $0 \le \alpha < \epsilon$.
	\end{lem}
	
	\begin{proof}
		Возьмём $\epsilon > 0$. Поле нормалей непрерывно $\Rightarrow$ равномерно непрерывно $\Rightarrow ~ \exists \delta > 0$, такое, что $\forall \Phi$ размера меньше $\delta, ~ \forall M_1, M_2 \in \widehat{\Phi} : \\ |\vec{n}(M_1) - \vec{n}(M_2)| < \sqrt{2 \epsilon}$. 
		Обозначим $\alpha = \dfrac{1}{2}(\vec{n}(M_1)-\vec{n}(M_2))^2$, тогда $0 \le \alpha < \epsilon$ и \\ 
		$\alpha = \dfrac{1}{2} \bigg( \underbrace{\|\vec{n}(M_1)\|^2}_{1} - 2 \underbrace{(\vec{n}(M_1), \vec{n}(M_2)}_{\cos \gamma}) + \underbrace{\|\vec{n}(M_2)\|^2}_{1} \bigg) = 1 - \cos \gamma$
	\end{proof}
	% заТеХано
\end{document}
